% Created 2021-12-06 Mo 20:09
% Intended LaTeX compiler: pdflatex
\documentclass[11pt]{article}
\usepackage[utf8]{inputenc}
\usepackage[T1]{fontenc}
\usepackage{graphicx}
\usepackage{grffile}
\usepackage{longtable}
\usepackage{wrapfig}
\usepackage{rotating}
\usepackage[normalem]{ulem}
\usepackage{amsmath}
\usepackage{textcomp}
\usepackage{amssymb}
\usepackage{capt-of}
\usepackage{hyperref}
\author{Yong}
\date{\today}
\title{}
\hypersetup{
 pdfauthor={Yong},
 pdftitle={},
 pdfkeywords={},
 pdfsubject={},
 pdfcreator={Emacs 26.3 (Org mode 9.4.6)}, 
 pdflang={English}}
\begin{document}

\tableofcontents

\section{Getting the source}
\label{sec:org9200199}
The third party libraries (such as, EES, CTP \ldots{}) are integrated into the source base
as git submodule.

The command to clone the whole source base including ThirdParty libraries is:
\emph{git clone --recurse-submodules -b <branch\_name> <repo\_address>}

\section{CMake}
\label{sec:orge7b3446}
The preferred build system is \textbf{CMake} by \emph{cmake -DCMAKE\_INSTALL\_PPREFIX=<XXX> -DBoost\_ROOT=<XXX>}. 
\subsection{Pre-requsites}
\label{sec:org0c4a7f6}
\subsubsection{\emph{python2.7}}
\label{sec:org5d14f57}
\subsubsection{\emph{boost 1.72.00}}
\label{sec:org0d8594e}

\begin{enumerate}
\item checkout version 1.72.0 in the git repo:
\begin{verbatim}
git checkout -f -b boost-1.72.0 boost-1.72.0
rm .git/modules/disjoint_sets
git submodule update --init
\end{verbatim}
\item build and install boost (\emph{boost\_python} should be built against \emph{python2.7})
\begin{verbatim}
./bootstrap --prefix=/home/yong/boost/1.72.0 --with-python=python2
./b2 install
\end{verbatim}
\end{enumerate}

\subsubsection{\emph{CMake} latest (>=3.15)}
\label{sec:org48a2038}
\subsubsection{\emph{autotools} is needed to build libev}
\label{sec:org8fe2f4d}
\subsubsection{\emph{openssl}}
\label{sec:org888b899}

\subsection{Dependencies}
\label{sec:orgf151d03}
\subsubsection{External (i.e. source available)}
\label{sec:org56806c7}
\begin{enumerate}
\item log4cplus
\label{sec:org1947b1c}
\item libev
\label{sec:org5bdbca8}
\end{enumerate}
\subsubsection{ThirdParty (i.e. only binary library available)}
\label{sec:org19e84ff}
\begin{enumerate}
\item CTP
\label{sec:orgcee00c2}
\item EES
\label{sec:orga08fa43}
\item Xt
\label{sec:org3aed372}
\end{enumerate}

\subsection{Configure \& Build}
\label{sec:org6f03837}

Example snippet:
\begin{verbatim}
mkdir build && cd build
cmake -DCMAKE_INSTALL_PREFIX=install_prefix -DENABLE_LIBEV_BUILD=ON ../src
make -j8
make install
\end{verbatim}

\subsubsection{CMAKE\_INSTALL\_PREFIX}
\label{sec:org844251c}
The default installation prefix is \textbf{/home/jarvis} if user does not pass in customized \emph{CMAKE\_INSTALL\_RPEFXI}.

\subsubsection{ENABLE\_LIBEV\_BUILD}
\label{sec:org30db51b}
This option controls whether building library \emph{libev} in the cmake configuration stage.
The default value is \textbf{ON}.
\emph{libev} should at least be built once.
Normally, in the first time of cmake configuration, the option should be \textbf{ON} so that \emph{libev} is built.
In the following times, the option can be set to \textbf{OFF} to save time of configuration.

\subsubsection{Boost\_ROOT}
\label{sec:org9405e74}
The directory points to the installation direcotry of \emph{boost} library.

\subsubsection{PYTHON2\_ROOT\_DIR}
\label{sec:org7e64e49}
The directory of python2 installation. This option normally can be ignored.
\emph{CMake} could find the correct python2 by itself if there is only one version of python2 installed.
If several versions of \emph{python2} installed, this option should point to the one which is used to complie \emph{boost} library.

\section{\emph{make} based build system}
\label{sec:orgdad9820}
\subsection{copy ThirdParty under src/ directory}
\label{sec:org991f5ea}
\subsection{edit \emph{src/make/head.mk}, change \textbf{boost\_DIR} and \textbf{PYTHON\_BIN} accordingly}
\label{sec:orge2d9e4b}
\subsection{invoke \emph{make} command}
\label{sec:org064d164}
\subsubsection{a new \emph{release} directory will be created with all targets inside}
\label{sec:org1f262cc}
\section{Caveants}
\label{sec:orgba5f68c}
\subsection{change libev.so.4.0.0 to libev.so}
\label{sec:org6ad7a96}
\subsubsection{only dynamic linkage success right now, to be investigated later}
\label{sec:org8bc6d6b}

\section{Running the servers}
\label{sec:org2dd6709}
\subsection{\emph{TDEngine} and \emph{MDEngine}}
\label{sec:orgbd9057f}
The most essential servers are the \emph{MDEngine} for the market quotes and \emph{TDEngine} for the trade transactions.

Cautions:
\begin{itemize}
\item running the servers need root privilege
\item \emph{TDEngine} needs to be invoked first and run successfully (for the base information), then \emph{MDEngine} could be invoked
\item Both engines are run as daemon by default
\item The configuration file used can be found in these scripts
\end{itemize}

\subsubsection{Method1}
\label{sec:org483cc5a}
Each server is an independent process, which has its own directory under the installation direcotry.
An script (under \$CMAKE\_INSTALL\_PREFIX\$/MDEngine/scripts/md\_svr.sh and \$CMAKE\_INSTALL\_PREFIX\$/TDEngine/scripts/td\_svr.sh)
is provided to facilitate the setup of the running environment of each server.
\begin{verbatim}
# TDEngine
cd $CMAKE_INSTALL_PREFIX$/TDEngine/scripts # or MDEngine direcotry
sudo ./td_svr.sh start/stop
\end{verbatim}

\subsubsection{Method2}
\label{sec:org132e6bf}
An overall script (under \$CMAKE\_INSTALL\_PREFIX\$/scripts/engine\_svr.sh) exist to correctly load the \emph{TDEngine} and \emph{MDEngine} together.
\begin{verbatim}
cd CMAKE_INSTALL_PREFIX/MDEngine/scripts
sudo ./engine_svr.sh start/stop
\end{verbatim}
\begin{verbatim}
#include <stdio>

void main()
{
  std::cout << "Hello World!" << std::endl;
}
\end{verbatim}

\begin{verbatim}
cd ~
pwd
echo "Hello WOrld"
\end{verbatim}

\begin{verbatim}
import numpy as np
print('Hello World')
print(np.__version__)
\end{verbatim}

\begin{verbatim}
print(np.__version__)
\end{verbatim}

Inline python code like \texttt{1.21.4}

\begin{verbatim}
import time 
print(f"Hello, today's date is {time.ctime()}")
print('Two plus two is')
return 3+2
\end{verbatim}

\begin{verbatim}
whoami
hostname
\end{verbatim}

\begin{verbatim}
return x*x
\end{verbatim}

\begin{verbatim}
81
\end{verbatim}
\end{document}
